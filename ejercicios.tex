\documentclass[12pt]{article}
\usepackage[utf8]{inputenc}
\usepackage[spanish]{babel}
\usepackage{fancyhdr}
\usepackage{amsmath,amsfonts,amsthm,xcolor,amssymb,mathtools}
\pagenumbering{gobble}
\usepackage{geometry}\geometry{margin=1in}
\usepackage{hyperref}
\usepackage{amssymb}
\usepackage{eufrak}
\usepackage{tikz-cd}
\usepackage{tcolorbox}
\usepackage{enumitem}% http://ctan.org/pkg/enumitem
\setlist[itemize]{noitemsep, topsep=0pt}
\usepackage{verbatim} %Comentar una sección a la vez, más cómodo que usar %

%Tipografía fancy
\usepackage{mathpazo}

%%%%%%%%%%%%%  Teoremas  %%%%%%%%%%%%%%%%%

\theoremstyle{plain}
\newtheorem{teo}{Teorema}
\newtheorem{propo}{Proposición}
\newtheorem{lema}[teo]{Lema}
\newtheorem{ej}{Ejercicio}


%Definiciones
\theoremstyle{definition}
\newtheorem*{definition}{Definition}
\newtheorem*{sol}{Solución}
\newtheorem*{lemm}{Lema}

% Observaciones
\theoremstyle{remark}
\newtheorem{obs}{Observación}

%Para poner una afirmación.
\newenvironment{claim}[1]{\par\noindent\underline{Afirmación:}\space#1}{}
\newcommand\coker{\text{coker} \hspace{0.1cm} \phi}
\newcommand\im{\text{im} \hspace{0.1cm} \phi}
\newcommand\NN{\mathbb{N}}
\newcommand\ZZ{\mathbb{Z}}
\newcommand\RR{\mathbb{R}}
\newcommand\CC{\mathbb{C}}
\newcommand\B{\mathscr{B}}
\newcommand\A{\mathscr{A}}
\newcommand\G{\mathscr{G}}
\newcommand\X{\mathfrak{X}}
\newcommand\g{\mathfrak{g}}
\DeclarePairedDelimiter\abs{\lvert}{\rvert}
\DeclarePairedDelimiter\norm{\lVert}{\rVert}
\DeclarePairedDelimiter\ip{\langle}{\rangle}
\DeclareMathOperator{\id}{id}
\DeclareMathOperator{\GL}{GL}
\DeclareMathOperator{\SL}{SL}
\DeclareMathOperator{\Diff}{Diff}
\let\O\relax
\DeclareMathOperator{\O}{O}
\DeclareMathOperator{\U}{U}
\newcommand\eps{\varepsilon}
\newcommand\ol{\overline}
\newcommand\fder[2][]{\frac{\partial^{#1}}{\partial #2}}
\newcommand\rder[2]{\frac{\partial{#1}}{\partial #2}}
\newcommand\dd{\mathrm{d}}
\newcommand{\red}{\textcolor{red}}
\newcommand\smatrix[1]{\left(\begin{smallmatrix}#1\end{smallmatrix}\right)}

%opening
\usepackage{fancyhdr}
\pagestyle{fancy}
\lhead{Ejercicios de teoría de grupos.} % Left Header
\rhead{\thepage} % Right Header

\usepackage{subfiles} % Best loaded last in the preamble

\title{\color{red!55!black} Ejercicios de teoría de grupos.}
\author{Leopoldo Lerena.}
\date{1er Cuatrimestre de 2021.}

\begin{document}
\maketitle

\medskip
\begin{tcolorbox}[colback=teal!25!white,colframe=teal!75!black]
\begin{ej}
	Dadas representaciones de un grupo $\rho: G \to GL(V), \psi: G \to GL(V)$ ambas son equivalentes sii vistos como $\CC[G]$ módulos son isomorfos.
\end{ej}	
\end{tcolorbox}
\medskip

Dos representaciones $\rho: G \to GL(V), \  \psi: G \to GL(V)$ son equivalentes si el siguiente diagrama conmuta para todo $g \in G$,

\begin{equation}\label{diagrama-equivalentes}
\begin{tikzcd}
\Bbb V  \arrow{d}{\rho_g} \arrow{r}{T}   &  \Bbb W \arrow{d}{\psi_g}  \\
\Bbb V \arrow{r}{T}     &  \Bbb W .
\end{tikzcd}
\end{equation}
\medskip
\begin{sol}	
	Supongamos que tenemos un isomorfismo $S: \Bbb V \to \Bbb W$ como $\CC G$ módulos. En particular es equivariante con respecto a la acción de $G$, es decir que 
	\[
	S(g \cdot v) = g \cdot S(v) \ \  \text{para todo} \ v \in \Bbb V \ \text{y para todo} \ g \in G
	\]
	traduciendo esto dado que $G$ actua por medio de automorfismos del espacio vectorial tenemos 
	\[
	S \rho_g (v) = \psi_g S(v)
	\]
	y esto es que el diagrama \ref{diagrama-equivalentes} conmute.  De esta manera como $\CC \subseteq \CC G$ tenemos que en particular restringiendo es un morfismo de $\CC$ módulos. Aparte que sabemos que $S$ es un isomorfismo por lo que queda concluído que nuestras representaciones son equivalentes.
	
	Supongamos que nuestras representaciones son equivalentes, en tal caso tenemos que existe un isomorfismo de espacios vectoriales $T: \Bbb V \to \Bbb W$ tal que hace conmutar el diagrama \ref{diagrama-equivalentes}
	
	Justamente esta condición representada por la conmutatividad del diagrama nos dice que el morfismo es equivariante con respecto a la acción de $G$ en los espacios vectoriales $\Bbb V, \Bbb W$. De esta manera vimos que es un isomorfismo como $\CC G$ módulos dado que es un isomorfismo lineal que respeta la acción de $G$.
\end{sol}


\begin{tcolorbox}[colback=teal!25!white,colframe=teal!75!black]
	\begin{ej}
		Los ideales del producto de dos álgebras $A \times B$ son de la forma $I \times J$ para $I \subset A$, $J \subset B$ ideales.
	\end{ej}	
\end{tcolorbox}
\medskip

\begin{sol}
	Sea un ideal $K$ del producto $A \times B$. Podemos mirar las primeras coordenadas o las segundas coordenadas de los elementos de este ideal. Afirmo que estos son los ideales $I,J$ que estamos buscando. Para eso primero veamos que se tratan de ideales. Sea $$I= \{ a\in A : \exists b \in B \ (a,b) \in K  \} $$ y esto vale porque sea $(a,b) \in K$ luego si lo multiplicamos a izquierda por algún elemento $(a',b')(a,b) \in K$ por lo que en particular tenemos que $a'a \in I$. Por la misma razón vemos que es cerrado con respecto a la suma de esta manera obtuvimos que $I$ es un ideal del álgebra $A$ y de manera idéntica tenemos que $J$ es un ideal de $B$.
	
	Debemos ver entonces que $K = I \times J$. Para esto notemos que al ser álgebras con unidad podemos considerar al elemento $(1,0) \in A \times B$ tal que si $(a,b') \in K$ luego $(a,0) \in K$ también y haciendo lo mismo para algún $b \in J$ genérico obtenemos que sumando estos elementos que $(a,b) \in K$. De esta manera vimos que cualquier elemento génerico de $I \times J$ está en $K$. Por como elegimos a los ideales $I,J$ la otra implicación sale directamente.

	
	
\end{sol}	
	
\end{document}	