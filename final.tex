\documentclass[aspectratio=169, 9pt]{beamer}
\usepackage[utf8]{inputenc}
\usepackage[spanish]{babel}

\usepackage{xcolor}
\usepackage{amssymb}
\usepackage{pifont}
\newcommand{\xmark}{\ding{55}}
\usetheme[progressbar=frametitle]{metropolis}
\usepackage{appendixnumberbeamer}
\usepackage{hyperref}
\usepackage{eufrak}
\usepackage{tikz-cd}
\usepackage{tcolorbox}
\usepackage{enumitem}% http://ctan.org/pkg/enumitem
	\definecolor{ao(english)}{rgb}{0.0, 0.5, 0.0}

\setbeamercolor{background canvas}{bg=white}
\usepackage{multicol}
\usepackage{chronology}


\usepackage{booktabs}
\usepackage[scale=2]{ccicons}

\usepackage{pgfplots}
\usepgfplotslibrary{dateplot}

\definecolor{amber}{rgb}{1.0, 0.49, 0.0}
\setbeamercolor{progress bar}{fg=amber,bg=alerted text.fg!50!black!10}

\makeatletter
\setlength{\metropolis@titleseparator@linewidth}{2pt}
\setlength{\metropolis@progressonsectionpage@linewidth}{2pt}
\setlength{\metropolis@progressinheadfoot@linewidth}{2pt}
\makeatother

\definecolor{ultramarine}{RGB}{81, 131, 232} 
\setbeamercolor{frametitle}{bg= ultramarine}

\usepackage{framed}
\definecolor{shadecolor}{gray}{0.9}

\renewcommand\qedsymbol{\textcolor{orange}{$\blacksquare$}}

\usepackage[font={footnotesize}]{caption}

\usepackage{xspace}
\newcommand{\themename}{\textbf{\textsc{metropolis}}\xspace}

\title{Multiplicadores de Schur y teorema de Schur.}
\subtitle{Final de teoría de grupos.}
% \date{\today}
\date{}
\author{Leopoldo Lerena}
\institute{Universidad de Buenos Aires}
% \titlegraphic{\hfill\includegraphics[height=1.5cm]{logo.pdf}}

\begin{document}

\maketitle

%\begin{frame}{Índice}
%  \setbeamertemplate{section in toc}[sections numbered]
%  \tableofcontents%[hideallsubsections]
%\end{frame}

\section[Extensiones centrales]{Extensiones centrales.}

\begin{frame}[fragile]{Definiciones básicas.}



\bigskip

Una \alert{extensión} de un grupo $K$ por otro grupo $Q$ es un grupo $G$ tal que encaja en la siguiente sucesión exacta corta,

\begin{equation*}
    1\longrightarrow K\xrightarrow{\phantom{a}\iota\phantom{a}}G\xrightarrow{\phantom{a}\pi\phantom{a}}Q\longrightarrow 1
\end{equation*}
\pause
Diremos que la extensión es \alert{central} si aparte sucede que $\iota(K) \le ZG$.

\medskip

Un \alert{levantamiento} de un grupo $Q$ es una función $l:Q \to G$ tal que $\pi \circ l = id_Q$. Esto es elegir un transversal de $G/K \simeq Q$.
\pause
\bigskip
\begin{alertblock}{Objetivo.}
Clasificar todas las extensiones centrales de $Q$ por $K$.
\end{alertblock}
\end{frame}

\begin{frame}[fragile]{Cociclos}
Sea $G$ una extensión central de $Q$ por $K$ y sea $l$ algún levantamiento. Dados $x,y \in Q$ podemos definirnos una función $f:Q \times Q \to K$ tal que 
\begin{equation*}
l(xy) f(x,y) = l(x)l(y)    
\end{equation*} 

diremos que esta $f$ es un \alert{cociclo} de nuestra extensión $G$. Al conjunto de todas los cociclos lo denotaremos $Z^2(Q,K)$.
\medskip
\pause
\metroset{block=fill}
\begin{alertblock}{Proposición.}
	Una función $f:Q \times Q \to K$ es un cociclo sii satisface las siguientes dos propiedades para todos $x,y,z \in Q$,
	\begin{enumerate}
		\item \alert{Identidad del cociclo}: $f(y,z)f(xy,z)^{-1}f(x,yz) = f(x,y)$.
		\item $f(1,y)=1=f(x,1)$. 
	\end{enumerate}
\end{alertblock}
\pause
\begin{exampleblock}{Observación.}
 $Z^2(Q,K)$ resulta ser un grupo abeliano.
\end{exampleblock}
\end{frame}

\begin{frame}[fragile]{Cociclos}
El problema que tenemos es que la coclase que define una extensión depende del transversal elegido. 
%Es decir si tomamos transversales distintos $l,l':Q \to G$ le corresponden coclases distintas $f,f':Q \times Q \to K$.

\pause
\bigskip

\metroset{block=fill}
\begin{alertblock}{Proposición.}
	La diferencia de dos cociclos $f,g:Q \times Q \to K$ es una función $h:Q \to K$ tal que $h(1)=0$ y que hace valer la siguiente igualdad,
	\begin{equation*}
	f(x,y) g(x,y)^{-1} = h(x)h(xy)^{-1}h(y)
	\end{equation*}
\end{alertblock}



\end{frame}

\begin{frame}[fragile]{Cobordes.}

 Un \alert{coborde} es un cociclo $g:Q \times Q \to K$ para el cual existe alguna función $h:Q \to K$ con $h(1)=0$ que hace valer la siguiente igualdad,
 
 \begin{equation*}
     g(x,y) = h(y)h(xy)^{-1}h(x)
 \end{equation*}
 Al conjunto de los cobordes lo vamos a denotar $B^2(Q,K)$.
 
 \pause
 \medskip
 \begin{exampleblock}{Observación.}
 $B^2(Q,K)$ resulta ser un subgrupo de $Z^2(Q,K)$.
\end{exampleblock}
 
\end{frame}

\begin{frame}[fragile]{Segundo grupo de cohomología.}
Dado grupos $(Q,K)$ luego su \alert{segundo grupo de cohomología} es el grupo 
\begin{equation*}
    H^2(Q,K) = \dfrac{Z^2(Q,K)}{B^2(Q,K)}
\end{equation*}
\medskip
\pause
Dos extensiones centrales $G,G'$ de $K$ por $Q$ son equivalentes si existe algún isomorfismo  $\beta:G \to G'$ tal que hace conmutar a este diagrama,

\begin{equation*}
	\begin{tikzcd}
	1 \arrow[r] & K \arrow[r] \arrow[d, "id", no head] & G' \arrow[r]                   & Q \arrow[r] \arrow[d, phantom]       & 1 \\
	1 \arrow[r] & K \arrow[r]                          & G \arrow[u, "\beta"] \arrow[r] & Q \arrow[u, "id", no head] \arrow[r] & 1
	\end{tikzcd}
\end{equation*}



\pause
\metroset{block=fill}
\begin{alertblock}{Teorema}
Dos extensiones $G,G'$ son equivalentes como extensiones sii corresponden a la misma clase de $H^2(Q,K)$.
\end{alertblock}

\end{frame}


\section[Multiplicador de Schur.]{Multiplicador de Schur.}
\begin{frame}[fragile]{Multiplicador de Schur.}

Sea $Q$ un grupo. Su \alert{multiplicador de Schur} es el siguiente grupo $M(Q) = H^2(Q, \Bbb C^\times)$. \pause
\metroset{block=fill}
\begin{alertblock}{Teorema.}
	El multiplicador de Schur de un grupo finito es un grupo finito abeliano.
\end{alertblock} \pause
\textcolor{orange}{Demostración.}
\begin{itemize}
	\item Dada $f \in Z^2(Q, \Bbb C^\times)$ definimos $\sigma:Q \to \Bbb C^\times$ por medio de
	\begin{equation*}
		\sigma(x) = \prod_{z \in Q} f(x,z)
	\end{equation*} \pause
	tal que multiplicando la identidad del cociclo para todo $z$ obtenemos que vale la siguiente igualdad si $n=|Q|$,
	\begin{equation*}
		\sigma(y)\sigma(xy)^{-1}\sigma(x) = f(x,y)^n.
	\end{equation*} \pause
	\item Definimos $g(x,y)=f(x,y)h(y)h(xy)^{-1}h(x)$ donde $h(x)$ es una raíz n-ésima de $\sigma(x)^{-1}$ con $h(1)=1$. \pause Notar que $[g]=[f]$. Por el ítem anterior $g(x,y)^n=1$ por lo que cada $[f] \in M(Q)$ determina una función $g:Q \times Q \to C_n$ de manera que $M(Q)$ es finito.
\end{itemize}

\end{frame}

\section[Cubrimientos.]{Cubrimientos.}

\begin{frame}[fragile]{Motivación de teoría de representaciones.}

Una \alert{representación proyectiva} de un grupo $Q$ es algún morfismo de grupos $\rho:Q \to PGL_n(\Bbb C)$. \pause
\medskip

\begin{exampleblock}{Motivación histórica.}
Conseguir representaciones lineales a partir de representaciones proyectivas.
\end{exampleblock}

\bigskip
\pause

Si consideramos que una representación proyectiva $\rho:Q \to PGL_n(\Bbb C)$ es un morfismo de grupos, tomando levantados en $GL_n(\Bbb C)$ obtenemos que es una colección de operadores que cumplen que para ciertas funciones $T:Q \to GL_n (\Bbb C),  \ f_\rho: Q \times Q \to \Bbb C$.
\begin{equation*}
	T(x)T(y) = f_\rho(x,y)T(x,y)
\end{equation*}
\pause tenemos el siguiente diagrama que conmuta
\begin{equation*}
	\begin{tikzcd}
	& GL_n(\Bbb C) \arrow[d, "\pi"] \\
	Q \arrow[ru, "T"] \arrow[r, "\rho"] & PGL_n(\Bbb C)                
	\end{tikzcd}
\end{equation*}
\end{frame}

\begin{frame}[fragile]{Cubrimientos y levantados de proyecciones.}

Diremos que un grupo tiene la \alert{propiedad del levantado de representaciones proyectivas} si existe alguna extensión $U$ de $Q$ por algún otro grupo $K$, y cada vez que $f:U \to Q$ es un epimorfismo con núcleo $K$ y $\rho:Q \to PGL_n(\Bbb C)$ es una representación proyectiva existe alguna representación $\rho:U \to GL_n(\Bbb C)$ tal que hace conmutar al siguiente diagrama,

\begin{equation*}
\begin{tikzcd}
U \arrow[r, "f", two heads] \arrow[d, "\overline \rho"', dashed] & Q \arrow[d, "\rho"] \\
GL_n(\Bbb C) \arrow[r, "\pi", two heads]                         & PGL_n(\Bbb C)      
\end{tikzcd}
\end{equation*}

\pause
\medskip


Sea $Q$ un grupo. Una extensión central $U$ de $K$ por $Q$, para $K$ algún grupo abeliano, es un \alert{cubrimiento} de $Q$ si cumple
\begin{itemize}
	\item $K \le U'$.
	\item $U$ tiene la propiedad de levantamiento proyectiva.
\end{itemize}
\end{frame}

\begin{frame}[fragile]{Transgresiones.}
Dado un grupo abeliano $M$ denotaremos al grupo $M^* = \text{hom}(M, \Bbb C^\times)$.
\pause
\bigskip


Dada $G$ alguna extensión central de $K$ sobre $Q$ que sabemos está representada por alguna cocadena $e$ definimos un morfismo $\delta: K^{*} \to M(Q)$ por medio de $\delta(\phi) = [\phi \circ e].$ Este morfismo de grupos lo llamaremos la \alert{transgresión}.
\end{frame}

\begin{frame}[fragile]{Teorema I: epimorfismos.}
\metroset{block=fill}
\begin{alertblock}{Teorema.}
	La transgresión $\delta$ es un epimorfismo sii $G$ tiene la propiedad de levantamiento de representaciones proyectivas.
\end{alertblock}
\textcolor{orange}{Demostración.} Vamos a ver solo la ida.
\begin{itemize}
	\item Vamos a ver que $\rho:Q \to PGL_n(\Bbb C)$ se puede levantar si $[f_\rho] \in \text{im} \ \delta$.
	\pause
	\item Si $[f_\rho]$ está en la imagen entonces para todo $x,y \in Q$ existe $\phi \in K^*$ y $h:Q \to \Bbb C^\times$ tal que
	\begin{equation*}
		\phi \circ e (x,y) = f_\rho(x,y)h(y)h(xy)^{-1}h(x)
	\end{equation*}
	\pause
	\item Definimos entonces $\overline \rho:U \to GL_n(\Bbb C)$ por medio de
	\begin{equation*}
		\overline \rho(a,x) = \phi(a)h(x)T(x)
	\end{equation*}
	y verificamos que es un morfismo de grupos que hace valer la propiedad del levantamiento de representaciones proyectivas.
\end{itemize}
\end{frame}


\begin{frame}{Lemas de divisibilidad.}
	Un grupo $D$ es \alert{divisible} si para cada $x \in D$ y para cada $n \ge 2$ existe algún $g \in D$ tal que $g^n = x$ para todo $n \ge 2$. \pause
	\metroset{block=fill}
	\begin{alertblock}{Propiedad inyectiva, Baer.}
		Sea $D$ un grupo divisible, $A \le B$. Si $f:A \to D$ es un morfismo de grupos entonces $f$ puede extenderse a un morfismo de grupos $f:B \to D$.
	\end{alertblock}
	\metroset{block=fill}
	\begin{alertblock}{Teorema.}
	Sea $G$ es un grupo abeliano, $a \in G$, $S$ un subgrupo de $G$ con $a \notin S$. Entonces existe un morfismo $\phi:G \to \Bbb C^\times$ con $\phi(S)=0$ y $\phi(a) \neq 0$.
	\end{alertblock}
\end{frame}


\begin{frame}[fragile]{Teorema II: monomorfismos.}
\metroset{block=fill}
\begin{alertblock}{Teorema.}
La transgresión $\delta$ es un monorfismo sii $K \le U'$.
\end{alertblock}
\textcolor{orange}{Demostración.} Vamos a ver solo la ida.
\begin{itemize}
	\item Primero veamos que si $\phi \in K^*$ y $\phi(K \cap U')=1$ entonces $\delta(\phi)=1$.
	\pause
	\item Por el teorema del segundo isomorfismo sabemos que $K/(K\cap U') \simeq KU'/U'$. Definimos $\psi:KU'/U' \to \Bbb C^\times$ como $\psi(aU')=\phi(a)$ que queda bien definida por la suposición del primer item.
	\pause
	\item Por la divisibilidad de $\Bbb C^\times$ levantamos el morfismo a $\Psi: U/U' \to \Bbb C^\times$.
	\pause
	\item Definimos $h:Q \to \Bbb C^\times$ por medio de $\Psi((1,x)U')$ y verificamos que \begin{equation*}
		\phi \circ e(x,y) = h(y)h(xy)^{-1}h(x)
	\end{equation*}
	por lo tanto es un coborde y esto nos dice que $\delta(\phi) = 1$.
	\pause
	\item Finalmente si $K \cap U' < K$ luego usando el teorema anterior debe existe $\phi \in K^*, \phi \neq 1$ tal que $K\cap U' < \ker \phi$ pero esto contradice lo probado al comienzo. Por lo tanto $K = K \cap U'$ tal como queríamos ver.
\end{itemize}
\end{frame}


\section[Teorema de Schur.]{Teorema de Schur.}

\begin{frame}[fragile]{Lema previo.}
\metroset{block=fill}
\begin{alertblock}{Lema.}
Si $Q$ es un grupo finito entonces $B^2(Q, \Bbb C^\times)$ tiene un complemento finito $M$ dentro de $Z^2(Q,\Bbb C^\times)$ tal que $M \simeq M(Q)$.
\end{alertblock}
\textcolor{orange}{Demostración.}

Veamos que el grupo $B^2(Q, \Bbb C^\times)$ es divisible. \pause
Para eso tomemos alguna $f \in B^2(Q,\Bbb C^\times)$ tal que
\[
f(x,y) = h(x)h(xy)^{-1}h(y)
\]
\pause
luego como $\Bbb C^\times$ es divisible nos tomamos $k(x)$ raíz enésima con $k(1)=1$. Afirmamos que 
\[
g(x,y) = k(x)k(xy)^{-1}k(y)
\]
nos sirve. \pause Al ser divisible por el teorema antes visto tiene un complemento y $M \simeq M(Q)$ y por lo tanto es finito.
\end{frame}

\begin{frame}[fragile]{Teorema de Schur.}

\metroset{block=fill}
\begin{alertblock}{Teorema de Schur 1904.}
Todo grupo finito $Q$ tiene un cubrimiento $U$ que es una extensión central de $M(Q)$ por $Q$.
\end{alertblock}

\pause
\textcolor{orange}{Demostración.} 
\begin{itemize}
	\item Consideramos $M$ un complemento de $B^2(Q, \Bbb C^\times)$ como en el lema anterior. Sea $K = M^*$. \pause Nos vamos a construir un cociclo $s:Q \times Q \to M^*$ y a partir de ésta una extensión central. \pause
	\item Definimos $s(x,y):M \to \Bbb C^\times$ por medio de $f\mapsto f(x,y).$ Verificamos que $s \in Z^2(Q, \Bbb C^\times)$. \pause
	\item Por la correspondencia existe alguna extensión $U$ de $M^*$ por $Q$. Afirmamos que es un cubrimiento. \pause Para verificar esto debemos ver que la transgresión $\delta: M^{**} \to  M(Q)$ es un isomorfismo. \pause
	
\end{itemize}

\end{frame}
\begin{frame}[fragile]{Demostración Schur (parte 2/2).}
\begin{itemize}
	\item Como $M \simeq M^{*}$ porque ambos son abelianos finitos entonces tenemos que $|M^{**}|=|M(Q)|$ por lo que alcanza con ver que es un epimorfismo.
	\pause
	\item Dado $[f] \in M(Q)$ luego $f \in Z^2(Q, \Bbb C^\times) = B^2 \times M$ por lo que $f=bf'$ con $b \in B^2$ y $f' \in M$ tal que $[f]=[f']$. \pause Podemos definir $\phi:M^{*} \to \Bbb C^\times$ por medio de $\phi(\mu) = \mu(f')$. \pause De esta manera vemos que $\delta(\phi)$ por definición es $[\phi(s(x,y))] = [s(x,y) (f')] = [f'(x,y)]$. \pause
	\medskip
	
	Verificamos que $\delta(\phi) = [f'] = [f]$ por lo tanto es sobreyectiva.
	
	\item Como vimos que es un isomorfismo tenemos que la extensión $U$ es un cubrimiento de $M^* \simeq M(Q)$ por $Q$ tal como queríamos ver.
\end{itemize}
\end{frame}
\section{Bibliografía.}
\begin{frame}[fragile]{Bibliografía consultada.}
\begin{itemize}
	\item VENDRAMIN, Teoría de grupos.
	\item ROTMAN, An introduction to the theory of groups, 4th edition, 1995, Springer-Verlag.
\end{itemize}
\end{frame}
\end{document}
